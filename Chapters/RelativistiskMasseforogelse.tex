\chapter{Relativistisk Massefor�gelse \& Hastighedsaddition}
\label{sec:RelativistiskMasseforogelse}
\epigraph{Life is an experiment.}{A. Einstein}
\minitoc
\clearpage

Vi har i kapitel \ref{sec:LorentzTransformationen} set p� hvorledes vi kunne genereliserer Galilei transformationerne\index{transformation!Galilei transformation}, p� den generelle form som lever op til det relativistiske princip\index{relativitets princippet} kaldte vi dem for Lorentz transformationer\index{transformation!Lorentz transformation}. Vi skal nu se p� hastighedstransformation som f�lge af Lorentz transformationen. Vi skal nu anvende ligning (\ref{eq:Lx} - \ref{eq:Lt}) til at udlede en relativistisk version af Galilei hastigheds transformationen, se ligning (\ref{eq:Gv}).  

Vi betragter her udelukkende en-dimentionel bev�gelse langs x-aksen og anvender begrebet ``{\bf hastighed}'' som en kort form af ``\emph{x-komponenter af hastigheden}.'' Antag at p� tiden $dt$ vil en partikel bev�ge sig afstanden $dx$, som m�lt af observat�ren i systemet $S$. Vi opn�r samh�rende v�rdier for afstand $dx'$ og tid $dt'$ i systemet $S'$. Ved nu at tage differentialerne af ligning (\ref{eq:Lx}) og ligning (\ref{eq:Lt}) finder vi:
\begin{align}
	\notag dx' &= \gamma\cdot(dx-u\,dt)\\
	\notag dt'  &= \gamma\cdot(dt-\frac{u\,dx}{c^2})
\end{align}
Divideres disse to ligninger nu med hinanden s�ledes at vi f�r $dx'/dt'$ finder vi:
\begin{equation*}
	\frac{dx'}{dt'} = \frac{dx-u\,dt}{dt-\frac{u\,dx}{c^2}}
\end{equation*}	
og deles b�de t�ller og n�vner i ovenst�ende resultat nu med $dt$ finder vi f�lgende:
\begin{equation*}
	\frac{dx'}{dt'} = \frac{\frac{dx}{dt}-u}{1-\frac{u}{c^2}\frac{dx}{dt}}
\end{equation*}
Afslutningsvis anvender vi vores viden fra Galilei hastigheds transformationen, nemlig at $dx/dt$ er hastigheden $v_x$  i $S$ og at $dx'/dt'$ er hastigheden $v_x'$ i systemet $S'$. Dermed har vi vist den generelle relativistiske hastigheds transformation.
\begin{equation}
	\label{eq:Lv}\tag{L.5} v_x' = \frac{v_x -u}{1-\frac{uv_x}{c^2}}
\end{equation}

N�r $u$ og $v_x$ er meget mindre end $c$, vil n�vneren i ligning (\ref{eq:Lv}) g� mod 1, og vi vil g� mod et ikke relativistisk resultat hvor $v_x' = v_x -u$. Det modsatte ekstremum hvor $v_x = c$; finder vi derimod:
\begin{equation*}
	v_x' = \frac{c-u}{1-\frac{uc}{c^2}} = \frac{c\cdot(1-\frac{u}{c})}{1-\frac{u}{c}} = c
\end{equation*}

Dette faktum betyder at hvis noget bev�ger sig med hastigheden $v_x = c$ m�lt i systemet $S$ vil hastigheden $v_x' = c$ ogs� hvis den m�les i $S'$, uanset den relative bev�gelse mellem de to systemer. Dermed er ligning (\ref{eq:lv}) i overenstemmelse med Einsteins postulat om at ``\emph{lysets hastighed i vakuum er den samme i alle initialsystemer}''.

Det relativistiske princip\index{relativitets princippet} fort�ller os her at der ikke er nogen fundemental forskel p� de to initialsystemer $S$ og $S'$. Derfor m� udtrykket for $v_x$, udtrykt ved $v_x'$ have samme form som ligning (\ref{eq:Lv}), med $v_x$ og $v_x'$ byttet om og liges� med fortegnet p� $u$. G�r man dette finder man 
\begin{equation}
	\label{eq:Lv2}\tag{L.5.1}v_x = \frac{v_x' -u}{1+\frac{uv_x'}{c^2}}
\end{equation}
B�de ligning (\ref{eq:Lv}) og ligning (\ref{eq:Lv2}) er Lorentz hastigheds transformationer \index{transformation!Lorentz transformation}. for en-demintionel bev�gelse.

N�r $u$ er mindre end $c$, fort�ller Lorentz hastigheds transformationen os at et objekt som bev�ger sig med en fart mindre end $c$ i et initialsystem, vil det samme objekt altid have en far mindre end $c$ i \emph{alle andre} initialsystemer. Heraf kan vi konkludere at intet fysisk objekt kan bev�ge sig med fart lig med eller st�rre end lysets i vakuum, relativt til et \emph{vilk�rligt} initialsystem. Den relativistiske generelisering af energi\index{energi} og bev�gelsesm�ngde\index{bev�gelsesm�ngde}, som vi senere skal se p� vil yderligere underbygge denne konklusion.

\emph{\small Dette afsnit er basseret p� \cite{YF2005,Adams1997}.}
