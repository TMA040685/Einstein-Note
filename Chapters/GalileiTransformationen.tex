\chapter{Galilei Transformationen}
\label{sec:GalileiTransformationen}

Vi vil starte med at beskrive forskellige reference rammer. Disse kaldes initialsystemer

\section{\color{MGR}\fontspec[SizeFeature={Size=16pr}]{Neo Tech Std Light}\MakeUppercase{Initialsystemer:}}
\label{sec:initialsystemer}
I fysik anvender man i forbindelse med relativitetsteorien\index{relativitetsteori} begrebet \emph{initialsystemer}\index{initialsystem}. Vi vil nu ganske kort gennem g� hvad der karakteriserer et initialsystem. 

Inden for fysikken afvender vi betegnelsen \emph{initialsystem}\index{initialsystem} om et koordinatsystem\index{koordinatsystem}, hvor alle legemer som ikke er p�virket af ydre kr�fter\item{kraft}, og som derfor bev�ger sig med konstante hastigheder. Der findes uendeligt mange initialsystemer\index{initialsystem}, der bev�ger sig med konstante hastigheder. Det vil med andre ord sige, at et koordinatsystem\index{koordinatsystem} som indeholder en partikel $P$, skal opfylde Newtons f�rste lov\index{Newton!1. lov} for at det kan betegnes som et initialsystem\index{initialsystem}.

\begin{quote}
	``\emph{En partikel som ikke er p�virket af en ydre resulterende kraft vil enten ligge stille eller bev�ge sig med konstant hastighed langs en ret linje.}''
	\begin{flushright}
		Sir Issac Newton\index{Newton!Issac Newton}
	\end{flushright}
\end{quote}


Lad os se lidt mere p� de to postulater som den specielle relativitetsteori\index{relativitetsteori!speciel relativitetsteori} er baseret p�. Begge postulater beskriver hvad en observat�r i et initialsystem ser. Teorien er speciel da denne indvirker p� observat�rer i netop s�dan et specielt initialsystem.

\begin{align}
\tag{G.1}\label{eq:Gx}	x &= x'+ut\\
\tag{G.2}\label{eq:Gy}	y &= y'\\
\tag{G.3}\label{eq:Gz}	z &=z'\\
\tag{G.4}\label{eq:Gt}	t &= t'
\end{align}


\begin{equation}
	\label{eq:Gv}\tag{G.5} v_x = v_x'+u
\end{equation}